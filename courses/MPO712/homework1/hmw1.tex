\documentclass[12pt,a4paper]{article}
\usepackage[latin1]{inputenc}
\usepackage{amsmath}
\usepackage{amsfonts}
\usepackage{amssymb}
\usepackage{makeidx}
\usepackage{graphicx}
\usepackage{kpfonts}
\usepackage{fourier}
\usepackage[left=0.5cm,right=0.5cm,top=0.5cm,bottom=0.5cm]{geometry}

\begin{document}
\pagestyle{empty}

\begin{center}
\Huge \textbf{Attachments}
\end{center}

\vspace*{-0.5cm}

\begin{figure}[h]
\begin{center}
\includegraphics[width=11cm,keepaspectratio=true]{wind_field.png}\\
\end{center}\caption{Wind stress vectors field. The atmospheric gyres drive two large-scale oceanic gyres.}
\end{figure}

\begin{figure}[!h]
\begin{center}
\includegraphics[width=11cm,keepaspectratio=true]{sverdrup_transport.png}\\
\end{center}\caption{Sverdrup's transport of mass driven by the wind stress field $\tau (x,y)$. The negative (positive) colors indicate equatorward (poleward) transport of mass. The white region (no meridional transport) represent the zero-crossing of the wind stress curl.}
\end{figure}
\vfill


\clearpage
\newpage

\begin{figure}[p]
\begin{center}

\includegraphics[width=12cm,keepaspectratio=true]{sverdrup_streamfuncion.png}\\
\end{center}\caption{Streamfunction lines of the Sverdrup's Transport driven by the wind. The dashed (solid) lines indicate positive (negative) transport values in Sv. The red arrows represent the western boundary flow that closes the respective gyres.}
\end{figure}

\end{document}